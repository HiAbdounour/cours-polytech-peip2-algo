\documentclass[aspectratio=169]{beamer}

\usepackage[utf8]{inputenc}
\usepackage[french]{babel}
\usepackage[T1]{fontenc}
\usepackage{tikz}
\usepackage{tikz-cd}
\usepackage{listings}
\usepackage{lstautogobble}
\usepackage{tcolorbox}
\tcbuselibrary{listings}
\tcbuselibrary{listingsutf8}

\usetheme{metropolis}

\usepackage{macros}

\AtBeginSection[]{
  \begin{frame}
  \vfill
  \centering
  \begin{beamercolorbox}[sep=8pt,center]{title}
    \usebeamerfont{title}\insertsectionhead\par%
  \end{beamercolorbox}
  \vfill
  \end{frame}
}

\setbeamertemplate{itemize item}{{\usebeamercolor[fg]{alerted
      text}{${\scriptstyle \blacktriangleright}$}}}


\usepackage{xpatch}
\makeatletter
\newlength{\my@beamer@itemsepi}\setlength{\my@beamer@itemsepi}{3ex}
\newlength{\my@beamer@itemsepii}\setlength{\my@beamer@itemsepii}{1.5ex}
\newlength{\my@beamer@itemsepiii}\setlength{\my@beamer@itemsepiii}{1.5ex}
\newcommand{\my@beamer@setsep}{%
  \ifnum\@itemdepth=1\relax
    \setlength\itemsep{\my@beamer@itemsepi}
  \else
    \ifnum\@itemdepth=2\relax
      \setlength\itemsep{\my@beamer@itemsepii}
    \else
      \ifnum\@itemdepth=3\relax
        \setlength\itemsep{\my@beamer@itemsepiii}
      \fi\fi\fi}
\xpatchcmd{\itemize}
{\def\makelabel}{\my@beamer@setsep\def\makelabel}{}{}
\xpatchcmd{\beamer@enum@}
{\def\makelabel}{\my@beamer@setsep\def\makelabel}{}{}
\newcommand\setlistsep[3]{%
  \setlength{\my@beamer@itemsepi}{#1}%
  \setlength{\my@beamer@itemsepii}{#2}%
  \setlength{\my@beamer@itemsepiii}{#3}%
}
\makeatother

\setlistsep{6.5ex}{2ex}{2ex}

%%%
\definecolor{lbcolor}{rgb}{0.1,0.1,0.1}
\definecolor{commentcolor}{rgb}{0.4,0.4,0.4}
\definecolor{keywordcolor}{HTML}{531ab6}
\definecolor{stringcolor}{HTML}{005f5f}

\lstset{
  basicstyle=\small\ttfamily\color{black},
  commentstyle=\rmfamily\color{commentcolor},
  keywordstyle=\bfseries\color{keywordcolor},
  showspaces=false,
  showstringspaces=false,
  stringstyle=\color{stringcolor},
  tabsize=2,
}

\newtcblisting{slidelisting}{
      arc=5mm,
      top=0mm,
      bottom=0mm,
      left=0mm,
      right=0mm,
      boxrule=1pt,
      listing only,
      listing options={language=C++},
      width=\textwidth
    }
    %%%
    \newtcbinputlisting{\slideinputlisting}[2][0]{
      listing file = #2,
      size = small,
      arc=5mm,
      top=0mm,
      bottom=0mm,
      left=0mm,
      right=0mm,
      boxrule=1pt,
      listing only,
      listing options={language=C++, firstline=#1},
      width=\textwidth
}

\newcommand{\codeslide}[2][4]{
  \slideinputlisting[#1]{#2}
  \onslide<2>
  \slideinputlisting[0]{#2.res}}

\title{Algorithmique et complexité}
\subtitle{Polytech Paris-Saclay, PEIP 2, Informatique 3}
\author{Thibaut Benjamin}
\date{7 Novembre 2025}
\institute{Amphi 1}
\begin{document}

\maketitle

\begin{frame}
  \frametitle{\'Equipe d'enseignement}

  \begin{itemize}
  \item Amphi

    \textbf{Thibaut Benjamin}
    \href{mailto:thibaut.benjamin@universite-paris-saclay.fr}
    {\texttt{thibaut.benjamin@universite-paris-saclay.fr}}

  \item TD/TP

    \textbf{Thibaut Benjamin}

    \textbf{Melissa Larbi}

    \textbf{Ali Sahili}

    \textbf{Julien Rauch}
  \end{itemize}

  \alert{Nous sommes là pour vous, n'hésitez pas à venir vers nous lorsque vous
    avez des questions.}

\end{frame}

\begin{frame}
  \frametitle{Objectif du cours}

  \begin{itemize}
  \item Renforcer les compétences en programmation \Cpp
  \item Comprendre les bases théoriques le la complexité des algorithmes
  \item Développer une connaissance des algorithmes classique
  \item Apprendre à analyser la complexité d'un algorithme
  \end{itemize}

\end{frame}

\begin{frame}
  \frametitle{Organisation du cours}

  \begin{itemize}
  \item 4 Cours en Amphis
  \item 4 Séances de TD
  \item 4 Séances de TP
  \end{itemize}

\end{frame}


\begin{frame}
  \frametitle{\'Evaluations}

  \begin{itemize}
  \item Examen de 2h Vendredi 5 Décembre

    \alert{\textbf{Tous documents papiers autorisés}}

  \item 1 Séance de mini-contrôle noté en TD
  \end{itemize}

\end{frame}

\begin{frame}
  \frametitle{Ressource en ligne}

  Une adresse à ajouter à vos favoris! (cours, TD, TP, etc...)

  \url{https://thibautbenjamin.github.io/cours-polytech-peip2-algo}

  \vspace{1.5cm}
  \onslide<2->
  \alert{\textbf{Si vous le souhaitez, contribuez au document de cours!}}

  Venez me voir si vous ne savez pas comment vous y prendre.

\end{frame}

\begin{frame}
  \frametitle{Séance du jour}

  \begin{itemize}
  \item Comprendre ce que la complexité d'un algorithme représente
  \item Réviser les bases de la programmation en \Cpp
  \item Aborder la notion de tableau en \Cpp
  \end{itemize}

\end{frame}

\section{La complexité des algorithmes}

\begin{frame}
  \frametitle{Qu'est-ce qu'un algorithme?}

  \begin{definition}
    Un algorithme est une séquence d'intstructions \alert{élémentaires}
    permettant de réaliser une tache.
  \end{definition}

  \vspace{1cm}

  Pour ce cours, un algorithme est un programme en \Cpp


\end{frame}

\begin{frame}
  \frametitle{Un exemple: Les algorithme de tri}

  \begin{itemize}
  \item On dispose d'un paquet de données que l'on peut comparer entre elles
    pour savoir si laquelle est la plus grande.
  \item On souhaite trier ces données par ordre croissant
  \item De nombreux algorithmes pour faire cela (voir le prochain amphi)
  \item Une visualisation utile:
    \url{https://mszula.github.io/visual-sorting/}
  \end{itemize}


\end{frame}

\begin{frame}
  \frametitle{Visualisation des algorithmes de tri}
  \begin{itemize}
      \item rendez-vous sur:
        \url{https://mszula.github.io/visual-sorting/}

      \item regardez les tris suivants
        \begin{itemize}
        \item Tri par sélection (selection sort)
        \item Tri par insertion (insertion sort)
        \item Tri à bulles (bubble sort)
        \item Tri fusion (merge sort)
        \item Tri rapide (quicksort)
        \end{itemize}
  \end{itemize}
\end{frame}


\begin{frame}
  \frametitle{Qu'est-ce que la complexité d'un algorithme?}

  \begin{definition}
    La complexité d'un algorithme est le nombre d'opérations que cet algorithme
    effectue en fonction de la taille de ses entrées.
  \end{definition}

  \vspace{1cm}
  \alert{\textbf{Temps d'exécution}}: De manière générale, plus un algorithme a
  une complexité élevée, et plus il mettra de temps à s'exécuter.

\end{frame}


\begin{frame}
  \frametitle{Complexité asymptotique}
  \begin{itemize}
  \item Sur des entrées de petites tailles, avec un ordinateur moderne, les
    algorithmes s'exécuteront tous très rapidement, peu importe leur complexité.
  \item C'est pourquoi on regarde la complexité quand les entrées deviennent
    très grandes\footnote{En mathématiques, on dit que leur taille tend vers
      l'infini}. On parle de \alert{complexité asymptotique}.
  \item On cherchera en général à calculer une \alert{approximation} de la
    complexité.

    \alert{Si les entrées doublent de taille, est-ce qu'on va devoir attendre
      quelques secondes de plus, ou quelques années de plus?}
  \end{itemize}

\end{frame}


\section{Rappels de \texorpdfstring{\Cpp}{C++}}

\begin{frame}[fragile]
  \frametitle{Les fichiers \texorpdfstring{\Cpp}{C++}}

  \codeslide[0]{code/example1.cpp}
\end{frame}

\begin{frame}[fragile]
  \frametitle{Déclaration et affectation des variables}

\codeslide{code/example2.cpp}

\end{frame}

\begin{frame}[fragile]
  \frametitle{Exemple: échange de la valeur}

  \codeslide{code/example3.cpp}

\end{frame}

\begin{frame}[fragile]
  \frametitle{Les types de données}

  \begin{itemize}
  \item \lstinline{int} : Les nombres entiers positifs ou
    négatifs\footnote{Techniquement, compris entre deux bornes}
  \item \lstinline{char} : Les caractères
  \item \lstinline{bool} : Les booléens, c'est à dire \lstinline{true} ou \lstinline{false}.
  \end{itemize}
\end{frame}

\begin{frame}[fragile]
  \frametitle{Définir des fonctions}

\codeslide{code/example4.cpp}

\end{frame}

\begin{frame}[fragile]
  \frametitle{Passage par valeurs}

  \codeslide{code/example5.cpp}

\end{frame}


\begin{frame}[fragile]
  \frametitle{Passage par valeurs : Explication}

  \begin{tikzpicture}
\node (E) at (0,0) {echanger({\color{blue}{int x}}, {\color{green}{int y}})};
\draw (-.5,-.5) node[rectangle, draw=blue, thick] (vx) {\alt<4->{0}{\phantom{0}}};
\draw (-.9,-.5) node (x) {x\textsubscript{e}};
\draw (.8,-.5) node[rectangle, draw=green, thick] (vy) {\alt<4->{1}{\phantom{0}}};
\draw (.4,-.5) node (x) {y\textsubscript{e}};
\draw (-1.5,-1) node (start) {\{};
\draw (-.9,-2) node (z) {z\textsubscript{e}};
\draw (-.5,-2) node[rectangle, draw=blue, thick]  {\alt<5->{0}{\phantom{0}}};
\draw (-.1,-2) node {;};
\draw (.4,-2) node {x\textsubscript{e}};
\draw (.8, -2) node[rectangle, draw=green, thick] {\alt<5->{1}{\phantom{0}}};
\draw (1.2,-2) node {;};
\draw (1.6,-2) node {y\textsubscript{e}};
\draw (2, -2) node[rectangle, draw=blue, thick] {\alt<5->{0}{\phantom{0}}};
\draw (2.4,-2) node {;};
\draw (-1.5,-3) node (start) {\}};

\onslide<2->
\node (M) at (7,0) {main()};
\draw (6.5, -.5) node {\{};
\draw (7,-1) node {x};
\draw (7.4,-1) node[rectangle, draw=black, thick] {0};
\draw (7.8,-1) node {;};
\draw (8.2,-1) node {y};
\draw (8.6,-1) node[rectangle, draw=black, thick] {1};
\draw (9,-1) node {;};
\draw (8,-2) node (call) {echanger(x,y);};
\draw (8,-3) node {return 0;};
\draw (6.5,-3.5) node {\}};

\onslide<3->
\draw[->, thick] (call) -- (E) node[midway, above] {x\textsubscript{e}=0, y\textsubscript{e}=1};

\end{tikzpicture}

\end{frame}


\begin{frame}[fragile]
  \frametitle{Passage par référence}

  \codeslide{code/example6.cpp}

\end{frame}

\begin{frame}
  \frametitle{Les conditions}
  \codeslide{code/example7.cpp}

\end{frame}

\begin{frame}
  \frametitle{Les boucles}

  \codeslide{code/example8.cpp}
\end{frame}


\section{Les tableaux}

\begin{frame}
  \frametitle{Les tableaux: une structure de donnée}

  \begin{itemize}
  \item Les structures de données permettent de stocker des données de manière
    organisée. Il est important de choisir une structure de donnée adaptée aux
    opérations que l'on va avoir besoin de réaliser.

  \item Les tableaux permettent de stocker une collection ordonnée d'objets de
    même type.

  \item \textbf{Exemple}: un tableau d'\lstinline{int}:
    \begin{tabular}{|c|c|c|c|c|}
      \multicolumn{1}{c}{\footnotesize 0}
      & \multicolumn{1}{c}{\footnotesize 1}
      & \multicolumn{1}{c}{\footnotesize 2}
      & \multicolumn{1}{c}{\footnotesize 3}
      & \multicolumn{1}{c}{\footnotesize 4} \\
      \hline
      10 & -7 & 42 & 0 & 10 \\
      \hline
    \end{tabular}

    \textbf{\alert{Attention}}: On commence à compter les positions à partir de
    0.
  \end{itemize}

\end{frame}


\begin{frame}
  \frametitle{Avec les tableaux on doit}
  \begin{itemize}
  \item Garder la même taille tout au long du programme.
  \item Déclarer la taille du tableau au moment de l'initialisation.
  \item A la place de changer de taille, il faut déclarer un nouveau tableau
    avec la nouvelle taille et copier toutes le valeur de l'ancien tableau dans
    le nouveau.

    \alert{\textbf{Déconseillé}!}
  \end{itemize}

\end{frame}

\begin{frame}
  \frametitle{Avec les tableaux on peut}
  \begin{itemize}
  \item Accéder à un élément stocké à n'importe quelle position dans le tableau.
  \item Changer la valeur contenue à n'importe quelle position dans le tableau.
  \item\alert{\textbf{Attention!}} Dans le tableau signifie dans les bornes du
    tableau. Si un tableau est de longueur \(n\), cela veut dire n'importe
    quelle position entre \(0\) et \(n-1\).
  \end{itemize}

\end{frame}

\begin{frame}[fragile]
  \frametitle{Déclarer un tableau en \texorpdfstring{\Cpp}{C++}}
  \slideinputlisting[4]{code/example9.cpp}
\end{frame}

\begin{frame}[fragile]
  \frametitle{Initialisation d'un tableau}
  \slideinputlisting[4]{code/example10.cpp}
\end{frame}

\begin{frame}[fragile]
  \frametitle{Accès aux éléments d'un tableau}
  \codeslide{code/example11.cpp}

\end{frame}

\begin{frame}[fragile]
  \frametitle{Modification des éléments d'un tableau}
  \codeslide[10]{code/example12.cpp}

\end{frame}

\begin{frame}[fragile]
  \frametitle{Passer un tableau en argument d'une fonction}

  \codeslide[10]{code/example13.cpp}

\end{frame}

\begin{frame}[fragile]
  \frametitle{Passer un tableau en argument d'une fonction : Explications}

  \alert{La valeur d'un tableau est l'adresse de la zone en mémoire contenant ce
    tableau}


  \begin{tikzpicture}
\node (E) at (0,0) {echangerPrems(int a[])};
\draw (-.9,-.5) node (a) {a\textsubscript{e}};
\draw (-1.5,-1) node (start) {\{};
\draw (1.8,-2) node (z) {int z = a[0]; a[0] = a[1]; a[1] = z;};
\draw (-1.5,-3) node (start) {\}};

\onslide<1-3>
\draw (-.5,-.5) node[rectangle, draw=black, thick] (vx) {\phantom{0}};

\onslide<2->
\node (M) at (7,0) {main()};
\draw (6.5, -.5) node {\{};
\draw (7,-1) node {a};
\draw (7.4,-1) node[rectangle, draw=red, thick] (va) {\#};
\draw (7.8,-1) node {;};
\draw (8,-2) node (call) {echangerPrems(a);};
\draw (8,-3) node {return 0;};
\draw (6.5,-3.5) node {\}};

\node (tab) at (4,2) {
\(\begin{array}{|c|c|c|c|}
  \hline
  \alt<5->{1}{0} & \alt<5->{0}{1} & 2 & 3 \\
  \hline
\end{array}\)
};

\draw[->,thick,red] (va) -- (tab);

\onslide<3->
\draw[->, thick, dashed] (call) -- (E) node[midway, above] {a\textsubscript{e}=\#};

\onslide<4->
\draw (-.5,-.5) node[rectangle, draw=red, thick] (vx) {\#};
\draw[->,thick,red] (vx) -- (tab);
\end{tikzpicture}

\end{frame}


\begin{frame}
  \frametitle{Quand utiliser les tableaux}
  \begin{itemize}
  \item Il existe d'autres structures de données en \Cpp pour reprérenter des
    collections ordonnées d'éléments.
    \begin{itemize}
    \item Les listes
    \item Les vecteurs
    \end{itemize}
  \item Ces structures permettent de faire d'autres opérations de manière
    primitive.

    \alert{Par exemple, dans une liste il est facile d'ajouter un élément, mais
      accéder au élément en dernière position coute cher.}
  \item Il faut donc choisir la bonne structure en fonction des opérations que
    l'on veut faire dessus. On choisira les tableaux lorsqu'on a une
    \alert{taille fixe} et qu'on veut accéder rapidement à \alert{n'importe
      quelle} position.
  \end{itemize}

\end{frame}


\section{Récapitulatif}
\begin{frame}
  \frametitle{Algorithmes et complexité}

  Un algorithme est un programme, sa complexité est le nombre d'opération qu'il
  fait, plus la complexité d'un algorithme est élevée, plus il prendra de temps
  à s'exécuter.

\end{frame}

\begin{frame}
  \frametitle{La programmation en \texorpdfstring{\Cpp}{C++}}

  \begin{itemize}
  \item Variables, conditions et boucles
  \item Appels de fonction et passage par valeur
  \end{itemize}

\end{frame}

\begin{frame}
  \frametitle{Les tableaux en \texorpdfstring{\Cpp}{C++}}

  \begin{itemize}
  \item Permettent de stocker des collections ordonnées de nombres.
  \item Modifiés en place lorsqu'ils sont passés comme arguments aux fonctions.
  \item Taille fixe et accès à n'importe quelle position très rapide.
  \end{itemize}

\end{frame}


\end{document}


%%% Local Variables:
%%% mode: LaTeX
%%% TeX-master: t
%%% End:
